\chapter{曲线的局部理论}
%----------------------***----------------------
\section{习题}

% This is 40% cyan plus 60% black.
\begin{tcolorbox}
	[breakable, colback = Emerald!10, colframe = cyan!40!black, title = 题2.1]
	求下列曲线的弧长与曲率:

(1) $y=ax^2$;

(2) $\frac{x^2}{a^2}+\frac{y^2}{b^2}=1$;

(3) $\boldsymbol{r}(t) = (a \cosh t, b \sinh t)$;

(4) $\boldsymbol{r}(t) = (t, a \cosh \frac{t}{a})$ $(a>0)$.
\end{tcolorbox}

% This is 40% cyan plus 60% green.
\begin{tcolorbox}
	[breakable, colback = Emerald!10, colframe = cyan!40!green, title = 解2.1]
	(1)
	可设$r(t)=(t,at^2)$,
	于是
	\begin{equation*}
		\begin{split}
			s(t)
			&= \int_{0}^{t}|\boldsymbol{r}'(u)|du \\
			&= \int \sqrt{1+4a^2u^2}du \\
			&= \frac{1}{2}t\sqrt{1+4a^{2}t^2} + \frac{1}{2a}\arcsinh 2at
		\end{split}
	\end{equation*}
	另一方面有
	\begin{equation*}
		\frac{ds}{dt} = \sqrt{1+4a^2t^2}
	\end{equation*}
	于是
	\begin{equation*}
		\boldsymbol{t}=\frac{d\boldsymbol{r}}{dt}\frac{dt}{ds}
		=(1+4a^2t^2)^{-\frac{1}{2}}(1,2at)
	\end{equation*}
	由于$\boldsymbol{n}$是$\boldsymbol{t}$逆时针旋转$\frac{\pi}{2}$得到,
	即
	\begin{equation*}
		\boldsymbol{n}=(1+4a^2t^2)^{-\frac{1}{2}}(-2at,1)
	\end{equation*}
	而
	\begin{equation*}
		\frac{d\boldsymbol{t}}{ds}=\frac{d\boldsymbol{t}}{dt}\frac{dt}{ds}
			=(1+4a^2t^2)^{-2}(-4a^2t,2a)
	\end{equation*}
	根据$\kappa(t)\boldsymbol{n}=\frac{d\boldsymbol{t}}{ds}$有
	\begin{equation*}
		\kappa(t)=2a(1+4a^2t^2)^{-\frac{3}{2}}.
	\end{equation*}

(2)
取一种参数化表示为
$\boldsymbol{r}(t)=(a\cos t,b \sin t)$,
则
\begin{equation*}
	\begin{split}
		s(t)
		&= \int_{0}^{t}|\boldsymbol{r}'(u)|du \\
		&= \int_{0}^{t}\sqrt{a^2\sin^2u+b^2\cos^2u}du
	\end{split}
\end{equation*}
此时有
\begin{equation*}
	\frac{ds}{dt}=\sqrt{a^2\sin^2t+b^2\cos^2t}
\end{equation*}
因此
\begin{equation*}
	\boldsymbol{t}=\frac{d\boldsymbol{r}}{dt}\frac{dt}{ds}
	=(a^2\sin^2t+b^2\cos^2t)^{-\frac{1}{2}}(-a\sin t, b\cos t)
\end{equation*}
又因为$\boldsymbol{n}$为$\boldsymbol{t}$逆时针旋转$\frac{\pi}{2}$,
故
\begin{equation*}
	\boldsymbol{n}=(a^2\sin^2t+b^2\cos^2t)^{-\frac{1}{2}}(-b\cos t, -a\sin t)
\end{equation*}
而
\begin{equation*}
	\frac{d\boldsymbol{t}}{ds}=\frac{d\boldsymbol{t}}{dt}\frac{dt}{ds}
	=(a^2\sin^2t+b^2\cos^2t)^{-2}(-ab^2\cos t, -a^2b\sin t)
\end{equation*}
根据$\kappa(t)\boldsymbol{n}=\frac{d\boldsymbol{t}}{ds}$,有
\begin{equation*}
	\kappa(t)=ab(a^2\sin^2t+b^2\cos^2t)^{-\frac{3}{2}}.
\end{equation*}

(3)
由
\begin{equation*}
	\begin{split}
		s(t)
		&=\int_{0}^{t}|\boldsymbol{r}'(u)|du \\
		&=\int_{0}^{t}\sqrt{a^2\sinh^2u+b^2\cosh^2u}du
	\end{split}
\end{equation*}
注意到
\begin{equation*}
	\frac{ds}{dt}=\sqrt{a^2\sinh^2t+b^2\cosh^2t}
\end{equation*}
因此
\begin{equation*}
	\boldsymbol{t}=\frac{d\boldsymbol{r}}{dt}\frac{dt}{ds}
	=(a^2\sinh^2t+b^2\cosh^2t)^{-\frac{1}{2}}(a\sinh t,b \cosh t)
\end{equation*}
而由于$\boldsymbol{n}$为$\boldsymbol{t}$的逆时针旋转$\frac{\pi}{2}$,
故
\begin{equation*}
	\boldsymbol{n}=(a^2\sinh^2 t+b^2\cosh^2t)^{-\frac{1}{2}}(-b\cosh t, a\sinh t)
\end{equation*}
此时
\begin{equation*}
	\frac{d\boldsymbol{t}}{ds}=\frac{d\boldsymbol{t}}{dt}\frac{dt}{ds}
	=(a^2\sinh^2t+b^2\cosh^2t)^{-2}(ab^2\cosh t,-a^2b\sinh t)
\end{equation*}
由$\kappa (t)\boldsymbol{n}=\frac{d\boldsymbol{t}}{ds}$,
有曲率
\begin{equation*}
	\kappa(t) = -ab(a^2\sinh^2t+b^2\cosh^2t)^{-\frac{3}{2}}.
\end{equation*}

(4)
由
\begin{equation*}
	\begin{split}
		s(t)
		&= \int_{0}^{t}|\boldsymbol{r}'(u)|du \\
		&= \int_{0}^{t}\sqrt{1+\sinh^2\frac{u}{a}}du \\
		&= a\sinh \frac{t}{a}
	\end{split}
\end{equation*}
注意到
\begin{equation*}
	\frac{ds}{dt} = \sqrt{1+\sinh^2\frac{t}{a}}
\end{equation*}
因此
\begin{equation}
	\boldsymbol{t}=\frac{d\boldsymbol{t}}{dt}\frac{dt}{ds}
	=(1+\sinh^2\frac{t}{a})^{-\frac{1}{2}}(1,\sinh\frac{t}{a})
\end{equation}
因$\boldsymbol{n}$为$\boldsymbol{t}$逆时针旋转$\frac{\pi}{2}$,
故
\begin{equation*}
	\boldsymbol{n} = (1+\sin^{2}\frac{t}{a})^{-\frac{1}{2}}(-\sinh\frac{t}{a},1)
\end{equation*}
而
\begin{equation*}
	\frac{d\boldsymbol{t}}{ds}=\frac{d\boldsymbol{t}}{dt}\frac{dt}{ds}
	=(1+\sinh^2\frac{t}{a})^{-2}(-\frac{1}{a}\sinh\frac{t}{a}\cosh\frac{t}{a},\frac{1}{a}\cosh\frac{t}{a})
\end{equation*}
由于$\kappa (t)\boldsymbol{n}=\frac{d\boldsymbol{t}}{ds}$,
则曲率
\begin{equation*}
	\kappa(t)=\frac{1}{a}\cosh\frac{t}{a}(1+\sinh^2\frac{t}{a})^{-\frac{3}{2}}.
\end{equation*}
\end{tcolorbox}

% This is 40% cyan plus 60% black.
\begin{tcolorbox}
	[breakable, colback = Emerald!10, colframe = cyan!40!black, title = 题2.2]
	设曲线$\boldsymbol{r}(t)=(x(t),y(t))$,
	证明它的曲率为
	\begin{equation*}
		\kappa(t)=\frac{x'(t)y''(t)-x''(t)y'(t)}{((x')^2+(y')^2)^{-\frac{3}{2}}}
	\end{equation*}
\end{tcolorbox}

% This is 40% cyan plus 60% green.
\begin{tcolorbox}
	[breakable, colback = Emerald!10, colframe = cyan!40!green, title = 证明2.2]
	由
	\begin{equation*}
		\frac{ds}{dt}=((x'(t))^2+(y'(t))^2)^{-\frac{1}{2}}
	\end{equation*}
	则
	\begin{equation*}
		\boldsymbol{t}=\frac{d\boldsymbol{r}}{dt}\frac{dt}{ds}
		=((x'(t))^2+(y'(t))^2)^{-\frac{1}{2}}(x'(t),y'(t))
	\end{equation*}
	由于$\boldsymbol{n}$是$\boldsymbol{t}$逆时针旋转$\frac{\pi}{2}$得到,
	故
	\begin{equation*}
		\boldsymbol{n}=((x'(t))^2+(y'(t))^2)^{-\frac{1}{2}}(-y'(t),x'(t))
	\end{equation*}
	注意到
	\begin{equation*}
		(x'(t)((x'(t))^2+(y'(t))^2)^{-\frac{1}{2}})_{t}
		=
		x'(t)(x'(t)y''(t)-x''(t)y'(t))((x'(t))^2+(y'(t))^2)^{-\frac{3}{2}}
	\end{equation*}
	\begin{equation*}
		(y'(t)((x'(t))^2+(y'(t))^2)^{-\frac{1}{2}})_{t}
		=
		y'(t)(y'(t)x''(t)-y''(t)x'(t))((x'(t))^2+(y'(t))^2)^{-\frac{3}{2}}
	\end{equation*}
	于是
	\begin{equation*}
		\frac{d\boldsymbol{t}}{ds}=\frac{d\boldsymbol{t}}{dt}\frac{dt}{ds}
		=((x'(t))^2+(y'(t))^2)^{-2}(x'(t)y''(t)-x''(t)y'(t))(-y'(t),x'(t))
	\end{equation*}
	根据$\kappa(t)\boldsymbol{n}=\frac{d\boldsymbol{t}}{ds}$,
	我们有
	\begin{equation*}
		\kappa(t)=(x'(t)y''(t)-x''(t)y'(t))((x'(t))^2+(y'(t))^2)^{-\frac{3}{2}}.
	\end{equation*}
\end{tcolorbox}
% This is 40% cyan plus 60% black.
\begin{tcolorbox}
	[breakable, colback = Emerald!10, colframe = cyan!40!black, title = 题2.3]
	设曲线$C$在极坐标$(r,\theta)$上的表示为$r=f(\theta)$,
	证明曲线$C$的曲率表达式为
	\begin{equation*}
		\kappa(\theta)=\frac{f^2(\theta)+2\left( \frac{df}{d\theta}\right)^2-f(\theta)\frac{d^2f}{d\theta^2}}{\left(f^2(\theta)+\left(\frac{df}{d\theta}\right)^2\right)^{\frac{3}{2}}}
	\end{equation*}
\end{tcolorbox}

% This is 40% cyan plus 60% green.
\begin{tcolorbox}
	[breakable, colback = Emerald!10, colframe = cyan!40!green, title = 证明2.3]
	令
	\begin{equation*}
		\begin{split}
			x(\theta) &= f(\theta) \cos \theta \\
			y(\theta) &= f(\theta) \sin \theta
		\end{split}
	\end{equation*}
	求导有
	\begin{equation*}
		\begin{split}
			x'(\theta) &= f'\cos\theta - f\sin\theta \\
			y'(\theta) &= f'\sin\theta + f\cos\theta
		\end{split}
	\end{equation*}
	\begin{equation*}
		\begin{split}
			x''(\theta) &= f''\cos\theta - 2f'\sin\theta -f\cos\theta \\
			y''(\theta) &= f''\sin\theta +2f'\cos\theta -f\sin\theta
		\end{split}
	\end{equation*}
	带入曲率的计算公式中有
	\begin{equation*}
		\begin{split}
			\kappa(\theta)
			&=\frac{x'(\theta)y''(\theta)-x''(\theta)y'(\theta)}{((x'(\theta))^{2}+(y'(\theta))^{2})^{\frac{3}{2}}} \\
			&=\frac{(f'\cos\theta - f\sin\theta)(f''\sin\theta +2f'\cos\theta -f\sin\theta)-(f''\cos\theta - 2f'\sin\theta -f\cos\theta)(f'\sin\theta + f\cos\theta)}{((f'\cos\theta - f\sin\theta)^2+(f'\sin\theta + f\cos\theta)^2)^{\frac{3}{2}}} \\
			&=\frac{f^2(\theta)+2\left( \frac{df}{d\theta}\right)^2-f(\theta)\frac{d^2f}{d\theta^2}}{\left(f^2(\theta)+\left(\frac{df}{d\theta}\right)^2\right)^{\frac{3}{2}}}
		\end{split}
	\end{equation*}
\end{tcolorbox}

% This is 40% cyan plus 60% black.
\begin{tcolorbox}
	[breakable, colback = Emerald!10, colframe = cyan!40!black, title = 题2.4]
	求下列曲线的曲率和挠率
	
	(1) $\boldsymbol{r}(t) = (a\cosh t, a\sinh t,bt)$ $(a>0)$
	
	(2) $\boldsymbol{r}(t) = (3t-t^2,3t^2,3t+t^2)$
	
	(3) $\boldsymbol{r}(t) = (a(1-\sin t),a(1-\cos t), bt)$ $(a>0)$
	
	(4) $\boldsymbol{r}(t) = (at,\sqrt{2}a\ln t, \frac{a}{t})$ $(a>0)$
\end{tcolorbox}

% This is 40% cyan plus 60% green.
\begin{tcolorbox}
	[breakable, colback = Emerald!10, colframe = cyan!40!green, title = 解2.4]
	(1)
	由
	\begin{equation*}
		\frac{ds}{dt}
		=\sqrt{a^2\sinh^2t+a^2\cosh^2t+b^2}
		=\sqrt{2a^2\cosh^2t-a^2+b^2}
	\end{equation*}
	知
	\begin{equation*}
		\boldsymbol{t}=\frac{d\boldsymbol{r}}{dt}\frac{dt}{ds}
		=(2a^2\cosh^2t-a^2+b^2)^{-\frac{1}{2}}(a\sinh t, a\cosh t, b)
	\end{equation*}
	故
	\begin{equation*}
		\frac{d\boldsymbol{t}}{ds}=\frac{d\boldsymbol{t}}{dt}\frac{dt}{ds}
		=(2a^2\cosh^2t-a^2+b^2)^{-2}(a(a^2+b^2)\cosh t, -a(a^2-b^2)\sinh t, -2a^2b\sinh t\cosh t)
	\end{equation*}
	则曲率为
	\begin{equation*}
		\kappa(t)=|\frac{d\boldsymbol{t}}{ds}|
		=a(2a^2\cosh^2t-a^2+b^2)^{-\frac{3}{2}}(2b^2\cosh^2t+a^2-b^2)^{\frac{1}{2}}
	\end{equation*}
	于是
	\begin{equation*}
		\begin{split}
			\boldsymbol{n}&=\frac{1}{\kappa(t)}\frac{d\boldsymbol{t}}{ds} \\
			&=(2a^2\cosh^2t-a^2+b^2)^{-\frac{1}{2}}(2b^2\cosh^2t+a^2-b^2)^{-\frac{1}{2}}((a^2+b^2)\cosh t,(b^2-a^2)\sinh t, -2ab\sinh t\cosh t)
		\end{split}
	\end{equation*}
	则
	\begin{equation*}
		\boldsymbol{b}(s)=\boldsymbol{t}\wedge \boldsymbol{n}=(2b^2\cosh^2t+a^2-b^2)^{-\frac{1}{2}}(-b\sinh t,b\cosh t, -a)
	\end{equation*}
	注意到
	\begin{equation*}
		\begin{split}
			\dot{\boldsymbol{b}}(s)
			&=\frac{d\boldsymbol{b}(s)}{dt}\frac{dt}{ds} \\
			&=(2b^2\cosh^2t+a^2-b^2)^{-\frac{3}{2}}(2a^2\cosh^t-a^2+b^2)^{-\frac{1}{2}}(-b(a^2+b^2)\cosh t,b(a^2-b^2)\sinh t, 2ab^2\sinh t\cosh t)
		\end{split}
	\end{equation*}
	于是根据$\dot{\boldsymbol{b}}=-\tau(t)\boldsymbol{n}$有
	\begin{equation*}
		\tau(t)=\frac{b}{2b^2\cosh^2 t +a^2-b^2}
	\end{equation*}
	
	(2)
	由
	\begin{equation*}
		\begin{split}
			\frac{ds}{dt}
			&=\sqrt{(3-2t)^2+(6t)^2+(3+2t)^2} \\
			&=\sqrt{44t^2+18}
		\end{split}
	\end{equation*}
	则
	\begin{equation*}
		\boldsymbol{t}=\frac{d\boldsymbol{t}}{dt}\frac{dt}{ds}
		=(44t^2+18)^{-\frac{1}{2}}(3-2t,6t,3+2t)
	\end{equation*}
	故
	\begin{equation*}
		\frac{d\boldsymbol{t}}{ds}=\frac{d\boldsymbol{t}}{dt}\frac{dt}{ds}
		=(22t^2+9)^{-2}(-3(11t+3),27,-3(11t+3))
	\end{equation*}
	则曲率为
	\begin{equation*}
		\kappa(t)=|\frac{d\boldsymbol{t}}{ds}| = \frac{3\sqrt{11}}{(22t^2+9)^{\frac{3}{2}}}
	\end{equation*}
	于是法向量
	\begin{equation*}
		\boldsymbol{n}=\frac{1}{\kappa(t)}\frac{d\boldsymbol{t}}{ds}
		=\frac{1}{\sqrt{11}}(22t^2+9)^{-\frac{1}{2}}(-(11t+3),9,-(11t-3))
	\end{equation*}
	则副法向量为
	\begin{equation*}
		\boldsymbol{b}(s)=\boldsymbol{t}\wedge\boldsymbol{n}
		=\frac{1}{\sqrt{11}}(-\frac{3}{2},-1,\frac{3}{2})
	\end{equation*}
	而
	\begin{equation*}
		\dot{\boldsymbol{b}}=\frac{d\boldsymbol{b}}{dt}\frac{dt}{ds}=0
	\end{equation*}
	于是根据$\dot{\boldsymbol{b}}(s)=-\tau(s)\boldsymbol{n}(s)$,
	可知挠率
	\begin{equation*}
		\tau(s)=0.
	\end{equation*}
	
	(3)
	由
	\begin{equation*}
		\begin{split}
			\frac{ds}{dt}
			&=\sqrt{(-a\cos t)^2+(a\sin t)^2+b^2} \\
			&=\sqrt{a^2\cos^2t+a^2\sin^2t+b^2} \\
			&= \sqrt{a^2+b^2}
		\end{split}
	\end{equation*}
	记$c=\sqrt{a^2+b^2}$,
	于是
	\begin{equation*}
		\boldsymbol{t}=\frac{d\boldsymbol{r}}{dt}\frac{dt}{ds}
		=(-\frac{a}{c}\cos t,\frac{a}{c}\sin t,\frac{b}{c})
	\end{equation*}
	而由
	\begin{equation*}
		\frac{d\boldsymbol{t}}{ds}=\frac{d\boldsymbol{t}}{dt}\frac{dt}{ds}
		=(\frac{a}{c^2}\sin t,\frac{a}{c^2}\cos t,0)
	\end{equation*}
	故曲率
	\begin{equation*}
		\kappa(t)=|\frac{d\boldsymbol{t}}{ds}|=\frac{a}{c^2}
	\end{equation*}
	则法向量
	\begin{equation*}
		\boldsymbol{n}=\frac{1}{\kappa}\frac{d\boldsymbol{t}}{ds}=(\sin t,\cos t,0)
	\end{equation*}
	那么副法向量
	\begin{equation*}
		\boldsymbol{b}=\boldsymbol{t}\wedge\boldsymbol{n}
		=(-\frac{b}{c}\cos t)
	\end{equation*}
	
	(4)
	由
	\begin{equation*}
		\begin{split}
			\frac{ds}{dt}
			&=\sqrt{a^2+\left(\sqrt{2}a\frac{1}{t}\right)^2+\left(-\frac{a}{t}\right)^2} \\
			&=\sqrt{a^2+\frac{2a^2}{t^2}+\frac{a^2}{t^4}} \\
			&=a\sqrt{1+\frac{2}{t^2}+\frac{1}{t^4}} \\
			&=a(1+\frac{1}{t})
		\end{split}
	\end{equation*}
	于是
	\begin{equation*}
		\dot{\boldsymbol{t}}=\frac{d\boldsymbol{r}}{dt}\frac{dt}{ds}
		=\frac{t}{t+1}(1,\frac{\sqrt{2}}{t},-\frac{1}{t^2})
	\end{equation*}
	而由
	\begin{equation*}
		\frac{d\boldsymbol{t}}{ds}=\frac{d\boldsymbol{t}}{dt}\frac{dt}{ds}
		=\frac{t}{a(t+1)}(\frac{1}{(t+1)^2},-\frac{\sqrt{2}}{(t+1)^2},\frac{2t+1}{t^2(t+1)^2})
	\end{equation*}
	知
	\begin{equation*}
		\kappa(t)=|\frac{d\boldsymbol{t}}{ds}|=\frac{\sqrt{3t^2+(2+\frac{1}{t})^2}}{a(t+1)^3}
	\end{equation*}
	于是法向量
	\begin{equation*}
		\boldsymbol{n}=\frac{1}{\kappa(t)}\frac{d\boldsymbol{t}}{ds}=\frac{1}{\sqrt{3t^2+(2+\frac{1}{t})^2}}(t,-\sqrt{2}t,2+\frac{1}{t})
	\end{equation*}
	则副法向量
	\begin{equation*}
		\boldsymbol{b}=\boldsymbol{t}\wedge\boldsymbol{n}=\frac{1}{\sqrt{3t^2+(2+\frac{1}{t})^2}}(\sqrt{2},-2,-\sqrt{2}t)
	\end{equation*}
\end{tcolorbox}